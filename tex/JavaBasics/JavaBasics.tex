% =========================================================================== %
% Java Basics
% =========================================================================== %

\ifx\wholebook\relax\else
  \documentclass[a4paper,10pt,twoside]{book}
  %=============================================================================%
% Common things, settings, packages to include
%=============================================================================%

\usepackage{graphicx}
\usepackage{color}
\usepackage{makeidx}
\usepackage{ifpdf}
\usepackage{verbatim}

\ifpdf
  \usepackage{pdfpages}
\fi

% --------------------------------------------------------------------------- %
% Setting up stuff depeding on output format
% --------------------------------------------------------------------------- %

\ifpdf
  % special settings for pdf mode
  \usepackage[colorlinks]{hyperref}
  \usepackage{courier}
  
  \hypersetup{
    colorlinks,
    linkcolor=darkblue,
    citecolor=darkblue,
    pdftitle={The Eclipse Scout Book},
    pdfauthor={The Scout Community},
    pdfkeywords={Enterprise Framework, Eclipse, Java, Client-Side, Rich Client, Web Client, Mobile},
    pdfsubject={Computer Science}
  }
  
  \usepackage{caption}
  \captionsetup{margin=10pt,font=small,labelfont=bf}
\else
  % special stuff for html mode
  \usepackage[tex4ht]{hyperref}
\fi

% --------------------------------------------------------------------------- %
% Setting up printing range
% --------------------------------------------------------------------------- %

\parindent 1cm
\parskip 0.2cm
\topmargin 0.2cm
\oddsidemargin 1cm
\evensidemargin 0.5cm
\textwidth 15cm
\textheight 21cm

% --------------------------------------------------------------------------- %
% Setting up paragraph formatting
% --------------------------------------------------------------------------- %

\setlength{\parindent}{20pt} 

% --------------------------------------------------------------------------- %
% Setting up listings
% --------------------------------------------------------------------------- %

\usepackage{listings}
 
\definecolor{darkviolet}{rgb}{0.5,0,0.4}
\definecolor{darkgreen}{rgb}{0,0.4,0.2} 
\definecolor{darkblue}{rgb}{0.1,0.1,0.9}
\definecolor{darkgrey}{rgb}{0.5,0.5,0.5}
\definecolor{lightblue}{rgb}{0.4,0.4,1}
\definecolor{lightgray}{rgb}{0.97,0.97,0.97}

\renewcommand{\lstlistlistingname}{List of Listings}

% general settings
\lstset{
  basicstyle=\small\ttfamily,
  columns=fullflexible,
  breaklines=true,
  breakindent=10pt,
  prebreak=\mbox{{\color{blue}\tiny$\searrow$}},
  postbreak=\mbox{{\color{blue}\tiny$\rightarrow$}},
  showstringspaces=false,
  backgroundcolor=\color{lightgray}
}

% settings for xml files
\lstdefinelanguage{xml}
{
  commentstyle=\color{darkgrey}\upshape,
  morestring=[b]",
  morestring=[s]{>}{<},
  morecomment=[s]{<?}{?>},
  stringstyle=\color{black},
  identifierstyle=\color{darkblue},
  keywordstyle=\color{cyan},
  morekeywords={xmlns,name,point,factory,class}% list your attributes here
}

% settings for ini files
\lstdefinelanguage{ini}
{
  morecomment=[f][\color{darkgrey}\upshape][0]\#, % # is comment iff it's the first char on the line
  stringstyle=\color{black}
}

% settings for console output
\lstdefinelanguage{console}
{
  morecomment=[l]{C:},
  commentstyle=\color{darkblue}
}

% default settings (for java files)
\lstset{
  language=Java,
  emphstyle=\color{red}\bfseries,
  keywordstyle=\color{darkviolet}\bfseries,
  commentstyle=\color{darkgreen},
  morecomment=[s][\color{lightblue}]{/**}{*/},
  stringstyle=\color{darkblue},
}

% --------------------------------------------------------------------------- %
% eclipse stuff macros
% --------------------------------------------------------------------------- %
\newcommand{\menu}[1]{\textsc{#1} menu}
\newcommand{\contextmenu}[1]{\textsc{#1} context menu}
\newcommand{\button}[1]{\textsc{#1} button}
\newcommand{\tab}[1]{\textsc{#1} tab}
\newcommand{\icon}[1]{\textsc{#1} icon}
\newcommand{\wizard}[1]{\textit{#1} wizard}
\newcommand{\field}[1]{\textit{#1} field}
\newcommand{\node}[1]{\textit{#1} node}
\newcommand{\folder}[1]{\textit{#1} folder}
\newcommand{\element}[1]{\textit{#1}}
\newcommand{\java}[1]{\texttt{#1}}
\newcommand{\filename}[1]{\texttt{#1}}

% --------------------------------------------------------------------------- %
% cross reference macros
% --------------------------------------------------------------------------- %
\newcommand{\prtlabel}[1]{\label{prt:#1}}
\newcommand{\applabel}[1]{\label{apx:#1}}
\newcommand{\chalabel}[1]{\label{cha:#1}}
\newcommand{\seclabel}[1]{\label{sec:#1}}
\newcommand{\lstlabel}[1]{lst:#1}
\newcommand{\figlabel}[1]{\label{fig:#1}}
\newcommand{\tablabel}[1]{\label{tab:#1}}

\newcommand{\prtref}[1]{Part~\ref{prt:#1}}
\newcommand{\appref}[1]{Appendix~\ref{apx:#1}}
\newcommand{\charef}[1]{Chapter~\ref{cha:#1}}
\newcommand{\secref}[1]{Section~\ref{sec:#1}}
\newcommand{\lstref}[1]{Listing~\ref{lst:#1}}
\newcommand{\figref}[1]{Figure~\ref{fig:#1}}
\newcommand{\tabref}[1]{Table~\ref{tab:#1}}

% --------------------------------------------------------------------------- %
% graphics paths
% --------------------------------------------------------------------------- %
\graphicspath{
  {figures/}
  {Introduction/figures/}
  {ScoutInstallation/figures/}  
}

%=============================================================================%

  \pagestyle{headings}
  \graphicspath{{figures/} {../figures/}}
  \begin{document}
  \sloppy
\fi


% --------------------------------------------------------------------------- %
\chapter{Java Basics}
\applabel{java_basics}

% --------------------------------------------------------------------------- %
\section{Java SE Basics}
\applabel{javase_basics}

\fbox{
  \parbox{12cm}{
    Section waiting for contribution (2'000-3'000 words)
	
    The goal of this section is to provide the reader with a solid overview of the non-trivial
    Java concepts relevant for scout applications and central aspects of the framework itself.
    The focus of this section is on the Java Standard Edition (Java SE).
    Where appropriate, provide links to high quality online material, that is likely to exist for at least the next year or two.
  }
}

\subsection{Learning Java}

To progam Scout applications you need to have a solid understanding of the Java language.
Scout will only work for you if you have achieved a certain proficiency level in Java. 

Luckily, free online tutorials to learn Java are offered in many places.
A good starting point is the official Java documentation 
site\footnote{Official online Java tutorial: \url{http://docs.oracle.com/javase/tutorial/}}.
If you prefer to work with video tutorials we recommend ``Eclipse and Java for Total 
Beginners''\footnote{Eclipse and Java for Total Beginners: \url{http://eclipsetutorial.sourceforge.net/totalbeginner.html}}, 
although the installation used is somewhat out of date.
As for printed books, we suggest to start with either ``Head First Java''\cite{batessierra05} or ``Thinking in Java''\cite{eckel06}.
Highly recommended but slightly more advanced is ``Effective Java''\cite{bloch08}.

To solve really tricky Java problems there is often no way around the Java 
specification\footnote{The Java Language Specification \url{http://docs.oracle.com/javase/specs/}} itself.
Just make sure to pick the right Java version for your context.

\subsection{Advanced Java SE Concepts}

  * say which non-trivial things are vital to good understanding
  * threading
  * generics
  * annotations

% --------------------------------------------------------------------------- %
\section{Java EE Basics}
\applabel{javaee_basics}

\fbox{
  \parbox{12cm}{
    Section waiting for contribution (2'000-5'000 words)
    
    The goal of this section is to provide the reader with a solid overview of the non-trivial
    Java enterprise concepts relevant for scout applications and central aspects of the framework itself. 
	The focus of this section is on the Java Enterprise Edition (Java EE)
    Where appropriate, provide links to high quality online material, that is likely to exist for at least the next year or two.
  }
}

needs text

  * maybe the same as for java foundation, maybe not
  * jaas
  * http comm
  * servlet
  * servlet filters


\lstinputlisting[
  label=\lstlabel{tinyservlet.web_xml},
  caption=The \java{index.html} start page for the tiny servlet application.,
  language=html,
  float
]
{../code/tinyservlet/index.html}

\lstinputlisting[
  label=\lstlabel{tinyservlet.web_xml},
  caption=The \java{web.xml} file of the tiny servlet application.,
  language=xml,
  float
]
{../code/tinyservlet/WEB-INF/web.xml}

\lstinputlisting[
  label=\lstlabel{tinyservlet.TinyServlet},
  caption=The complete \java{TinyServlet} source code.,
  float
]
{../code/tinyservlet/WEB-INF/sources/TinyServlet.java}

war file organisation: \url{http://documentation.progress.com/output/Iona/orbix/6.1/tutorials/fnb/dev_intro/j2ee_overview8.html}

% --------------------------------------------------------------------------- %

\ifx\wholebook\relax\else
   \begin{thebibliography}{99}
  \addcontentsline{toc}{chapter}{Bibliography}
  
  \bibitem{lamport} L. Lamport. {\bf \LaTeX \ A Document Preparation System}
    Addison-Wesley, California 1986.

\end{thebibliography}

   \end{document}
\fi

% =========================================================================== %
