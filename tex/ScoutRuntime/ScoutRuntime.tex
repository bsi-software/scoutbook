% =========================================================================== %
% Scout Runtime
% =========================================================================== %

\ifx\wholebook\relax\else
  \documentclass[a4paper,10pt,twoside]{book}
  %=============================================================================%
% Common things, settings, packages to include
%=============================================================================%

\usepackage{graphicx}
\usepackage{color}
\usepackage{makeidx}
\usepackage{ifpdf}
\usepackage{verbatim}

\ifpdf
  \usepackage{pdfpages}
\fi

% --------------------------------------------------------------------------- %
% Setting up stuff depeding on output format
% --------------------------------------------------------------------------- %

\ifpdf
  % special settings for pdf mode
  \usepackage[colorlinks]{hyperref}
  \usepackage{courier}
  
  \hypersetup{
    colorlinks,
    linkcolor=darkblue,
    citecolor=darkblue,
    pdftitle={The Eclipse Scout Book},
    pdfauthor={The Scout Community},
    pdfkeywords={Enterprise Framework, Eclipse, Java, Client-Side, Rich Client, Web Client, Mobile},
    pdfsubject={Computer Science}
  }
  
  \usepackage{caption}
  \captionsetup{margin=10pt,font=small,labelfont=bf}
\else
  % special stuff for html mode
  \usepackage[tex4ht]{hyperref}
\fi

% --------------------------------------------------------------------------- %
% Setting up printing range
% --------------------------------------------------------------------------- %

\parindent 1cm
\parskip 0.2cm
\topmargin 0.2cm
\oddsidemargin 1cm
\evensidemargin 0.5cm
\textwidth 15cm
\textheight 21cm

% --------------------------------------------------------------------------- %
% Setting up paragraph formatting
% --------------------------------------------------------------------------- %

\setlength{\parindent}{20pt} 

% --------------------------------------------------------------------------- %
% Setting up listings
% --------------------------------------------------------------------------- %

\usepackage{listings}
 
\definecolor{darkviolet}{rgb}{0.5,0,0.4}
\definecolor{darkgreen}{rgb}{0,0.4,0.2} 
\definecolor{darkblue}{rgb}{0.1,0.1,0.9}
\definecolor{darkgrey}{rgb}{0.5,0.5,0.5}
\definecolor{lightblue}{rgb}{0.4,0.4,1}
\definecolor{lightgray}{rgb}{0.97,0.97,0.97}

\renewcommand{\lstlistlistingname}{List of Listings}

% general settings
\lstset{
  basicstyle=\small\ttfamily,
  columns=fullflexible,
  breaklines=true,
  breakindent=10pt,
  prebreak=\mbox{{\color{blue}\tiny$\searrow$}},
  postbreak=\mbox{{\color{blue}\tiny$\rightarrow$}},
  showstringspaces=false,
  backgroundcolor=\color{lightgray}
}

% settings for xml files
\lstdefinelanguage{xml}
{
  commentstyle=\color{darkgrey}\upshape,
  morestring=[b]",
  morestring=[s]{>}{<},
  morecomment=[s]{<?}{?>},
  stringstyle=\color{black},
  identifierstyle=\color{darkblue},
  keywordstyle=\color{cyan},
  morekeywords={xmlns,name,point,factory,class}% list your attributes here
}

% settings for ini files
\lstdefinelanguage{ini}
{
  morecomment=[f][\color{darkgrey}\upshape][0]\#, % # is comment iff it's the first char on the line
  stringstyle=\color{black}
}

% settings for console output
\lstdefinelanguage{console}
{
  morecomment=[l]{C:},
  commentstyle=\color{darkblue}
}

% default settings (for java files)
\lstset{
  language=Java,
  emphstyle=\color{red}\bfseries,
  keywordstyle=\color{darkviolet}\bfseries,
  commentstyle=\color{darkgreen},
  morecomment=[s][\color{lightblue}]{/**}{*/},
  stringstyle=\color{darkblue},
}

% --------------------------------------------------------------------------- %
% eclipse stuff macros
% --------------------------------------------------------------------------- %
\newcommand{\menu}[1]{\textsc{#1} menu}
\newcommand{\contextmenu}[1]{\textsc{#1} context menu}
\newcommand{\button}[1]{\textsc{#1} button}
\newcommand{\tab}[1]{\textsc{#1} tab}
\newcommand{\icon}[1]{\textsc{#1} icon}
\newcommand{\wizard}[1]{\textit{#1} wizard}
\newcommand{\field}[1]{\textit{#1} field}
\newcommand{\node}[1]{\textit{#1} node}
\newcommand{\folder}[1]{\textit{#1} folder}
\newcommand{\element}[1]{\textit{#1}}
\newcommand{\java}[1]{\texttt{#1}}
\newcommand{\filename}[1]{\texttt{#1}}

% --------------------------------------------------------------------------- %
% cross reference macros
% --------------------------------------------------------------------------- %
\newcommand{\prtlabel}[1]{\label{prt:#1}}
\newcommand{\applabel}[1]{\label{apx:#1}}
\newcommand{\chalabel}[1]{\label{cha:#1}}
\newcommand{\seclabel}[1]{\label{sec:#1}}
\newcommand{\lstlabel}[1]{lst:#1}
\newcommand{\figlabel}[1]{\label{fig:#1}}
\newcommand{\tablabel}[1]{\label{tab:#1}}

\newcommand{\prtref}[1]{Part~\ref{prt:#1}}
\newcommand{\appref}[1]{Appendix~\ref{apx:#1}}
\newcommand{\charef}[1]{Chapter~\ref{cha:#1}}
\newcommand{\secref}[1]{Section~\ref{sec:#1}}
\newcommand{\lstref}[1]{Listing~\ref{lst:#1}}
\newcommand{\figref}[1]{Figure~\ref{fig:#1}}
\newcommand{\tabref}[1]{Table~\ref{tab:#1}}

% --------------------------------------------------------------------------- %
% graphics paths
% --------------------------------------------------------------------------- %
\graphicspath{
  {figures/}
  {Introduction/figures/}
  {ScoutInstallation/figures/}  
}

%=============================================================================%

  \pagestyle{headings}
  \graphicspath{{figures/} {../figures/}}
  \begin{document}
  \sloppy
\fi


% --------------------------------------------------------------------------- %
\chapter{Scout Runtime}
\chalabel{runtime}

One of the primary goals of the Scout runtime framework is to let the application developer focus on implementing business requirements. 
To support this goal, the Scout runtime defines an application model that covers a wide variety of recurring aspects of business applications.

Developing complex user interface, dealing with input validation, client server communication, handling of codes, permissions, user authentication, logging, and accessing data bases or backend systems using web services are all aspects that are covered by Scout.
On the other hand, business modeling, a persistence layer, role models for user authorisation are outside of the scope of the Scout runtime.
This is by design, as Scout does not cover domains already addressed by powerful and well established open source frameworks.

Where possible, Scout relies on the standards defined by the OSGi/Eclipse platform and a small subset of J2EE components. 
Typically, just a thin convenience layer is added by Scout. 
This helps the beginner to pick up Scout concepts more quickly and -- at the same time -- offers the necessary flexibility to developers that are already familiar with those technologies. 

In this chapter we first discuss the framework scope based on the integration of a Scout application in a typical enterprise setup.
Then, we describe the internal architecture of Scout applications and highlight the difference of the architecture for scenarios involving desktop clients and web/mobile clients.
As the multi-frontend support for Scout applications is one of the most important featues of the Scout application model a specific section is dedicated to this feature.
Finally, the basic building blocks of a Scout application -- the Scout server, the Scout client, and the client server communication -- are described individually.


% --------------------------------------------------------------------------- %
\section{Enterprise Integration}
\seclabel{enterprise_integration}

\begin{figure}
\includediagram{14cm}{scout_integration}
\caption{The recommended usage of the Scout framework for enterprise applications.}
\figlabel{scout_integration_enterprise}
\end{figure}

The scope of the Scout framework is best explained in the context of an enterprise setup as shown in \figref{scout_integration_enterprise}. 
From

not persistence, not business entity modeling, not ?
but user interaction handling, business rules, accessing existing services offered through enterprise service bus, transparent client server communication

% --------------------------------------------------------------------------- %
\section{Application Architecture}
\seclabel{application_architecture}
needs text

\noindent Existing Documentation
\begin{itemize}
  \item wiki \url{http://wiki.eclipse.org/Scout/Concepts}
\end{itemize}

\begin{figure}
\includediagram{14cm}{scout_app_architecture_desktop}
\caption{The architecture of a typical Scout client server application.
On the left side, a Scout desktop client is depicted.}
\figlabel{scout_app_architecture_desktop}
\end{figure}

Architecture including desktop clients.

\begin{figure}
\includediagram{14cm}{scout_app_architecture_web}
\caption{A typical Scout client server setup including web and mobile clients.}
\figlabel{scout_app_architecture_web}
\end{figure}

Architecture when wordking with Ajax server for web applications

% --------------------------------------------------------------------------- %
\section{Multi-Frontend Support}

two important aspects: 1) same app running on many devices 2) effective risk reduction strategy: no mixing of business logic an ui tech code, deciding for the 'wrong' framwork not so bad any more

% --------------------------------------------------------------------------- %
\section{The Scout Client}
needs text

 large collection of mature UI components.
 Scout supports various UI technologies out of the box. 
 Depending on your needs, you decide to build applications for
mobile devices, the browser or the desktop. Mobile and
web applications are based on Eclipse RAP. Desktop clients
are based on either Swing or SWT.

internationalization, link to packaging with client, shared, ui tech, framwork + other bundles. 


\noindent Existing Documentation
\begin{itemize}
  \item scout concept wiki: client plug-in \url{http://wiki.eclipse.org/Scout/Concepts/Separation_UI_and_GUI}
\end{itemize}

% --------------------------------------------------------------------------- %
\section{The Scout Server}

For seamless integration into a service-oriented IT architecture, Scout offers direct support for Web services based on
JAX-WS. To access relational databases other than Apache
Derby, connectors are also available for non-EPL compatible databases such as Oracle, MySql, PostgreSQL or DB2

link to packaging with server, shared, framwork + other bundles

\noindent Existing Documentation
\begin{itemize}
  \item scout concept wiki: server plug-in \url{http://wiki.eclipse.org/Scout/Concepts/Server_Plug-In}
  \item scout concept wiki: security \url{http://wiki.eclipse.org/Scout/Concepts/Security}
\end{itemize}

% --------------------------------------------------------------------------- %
\section{Communication and Shared Components}
needs text

service tunnel

Icons Class that contains as static members the icons that are available. 
Permissions are.
Enumerations are codes A CodeType is a structure to represent a tree key-code association. 
They are used in SmartField and SmartColumn. 
Texts is a convenience class to access the Text Provider Services used for the localization of the texts in the user interface. 


\noindent Existing Documentation
\begin{itemize}
  \item forum tech background questions \url{http://www.eclipse.org/forums/index.php/t/299623/}
  \item concept wiki communication \url{http://wiki.eclipse.org/Scout/Concepts/Communication}
  \item concept wiki shared \url{http://wiki.eclipse.org/Scout/Concepts/Shared_Plug-In}
\end{itemize}

same pointer in onedaytutorial and server


\ifx\wholebook\relax\else
   \begin{thebibliography}{99}
  \addcontentsline{toc}{chapter}{Bibliography}
  
  \bibitem{lamport} L. Lamport. {\bf \LaTeX \ A Document Preparation System}
    Addison-Wesley, California 1986.

\end{thebibliography}

   \end{document}
\fi

% =========================================================================== %
