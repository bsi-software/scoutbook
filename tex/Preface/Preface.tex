% =========================================================================== %
% Preface
% please make sure that this fits into two pages (max)
% =========================================================================== %

\ifx\wholebook\relax\else
  \documentclass[a4paper,10pt,twoside]{book}
  %=============================================================================%
% Common things, settings, packages to include
%=============================================================================%

\usepackage{graphicx}
\usepackage{color}
\usepackage{makeidx}
\usepackage{ifpdf}
\usepackage{verbatim}

\ifpdf
  \usepackage{pdfpages}
\fi

% --------------------------------------------------------------------------- %
% Setting up stuff depeding on output format
% --------------------------------------------------------------------------- %

\ifpdf
  % special settings for pdf mode
  \usepackage[colorlinks]{hyperref}
  \usepackage{courier}
  
  \hypersetup{
    colorlinks,
    linkcolor=darkblue,
    citecolor=darkblue,
    pdftitle={The Eclipse Scout Book},
    pdfauthor={The Scout Community},
    pdfkeywords={Enterprise Framework, Eclipse, Java, Client-Side, Rich Client, Web Client, Mobile},
    pdfsubject={Computer Science}
  }
  
  \usepackage{caption}
  \captionsetup{margin=10pt,font=small,labelfont=bf}
\else
  % special stuff for html mode
  \usepackage[tex4ht]{hyperref}
\fi

% --------------------------------------------------------------------------- %
% Setting up printing range
% --------------------------------------------------------------------------- %

\parindent 1cm
\parskip 0.2cm
\topmargin 0.2cm
\oddsidemargin 1cm
\evensidemargin 0.5cm
\textwidth 15cm
\textheight 21cm

% --------------------------------------------------------------------------- %
% Setting up paragraph formatting
% --------------------------------------------------------------------------- %

\setlength{\parindent}{20pt} 

% --------------------------------------------------------------------------- %
% Setting up listings
% --------------------------------------------------------------------------- %

\usepackage{listings}
 
\definecolor{darkviolet}{rgb}{0.5,0,0.4}
\definecolor{darkgreen}{rgb}{0,0.4,0.2} 
\definecolor{darkblue}{rgb}{0.1,0.1,0.9}
\definecolor{darkgrey}{rgb}{0.5,0.5,0.5}
\definecolor{lightblue}{rgb}{0.4,0.4,1}
\definecolor{lightgray}{rgb}{0.97,0.97,0.97}

\renewcommand{\lstlistlistingname}{List of Listings}

% general settings
\lstset{
  basicstyle=\small\ttfamily,
  columns=fullflexible,
  breaklines=true,
  breakindent=10pt,
  prebreak=\mbox{{\color{blue}\tiny$\searrow$}},
  postbreak=\mbox{{\color{blue}\tiny$\rightarrow$}},
  showstringspaces=false,
  backgroundcolor=\color{lightgray}
}

% settings for xml files
\lstdefinelanguage{xml}
{
  commentstyle=\color{darkgrey}\upshape,
  morestring=[b]",
  morestring=[s]{>}{<},
  morecomment=[s]{<?}{?>},
  stringstyle=\color{black},
  identifierstyle=\color{darkblue},
  keywordstyle=\color{cyan},
  morekeywords={xmlns,name,point,factory,class}% list your attributes here
}

% settings for ini files
\lstdefinelanguage{ini}
{
  morecomment=[f][\color{darkgrey}\upshape][0]\#, % # is comment iff it's the first char on the line
  stringstyle=\color{black}
}

% settings for console output
\lstdefinelanguage{console}
{
  morecomment=[l]{C:},
  commentstyle=\color{darkblue}
}

% default settings (for java files)
\lstset{
  language=Java,
  emphstyle=\color{red}\bfseries,
  keywordstyle=\color{darkviolet}\bfseries,
  commentstyle=\color{darkgreen},
  morecomment=[s][\color{lightblue}]{/**}{*/},
  stringstyle=\color{darkblue},
}

% --------------------------------------------------------------------------- %
% eclipse stuff macros
% --------------------------------------------------------------------------- %
\newcommand{\menu}[1]{\textsc{#1} menu}
\newcommand{\contextmenu}[1]{\textsc{#1} context menu}
\newcommand{\button}[1]{\textsc{#1} button}
\newcommand{\tab}[1]{\textsc{#1} tab}
\newcommand{\icon}[1]{\textsc{#1} icon}
\newcommand{\wizard}[1]{\textit{#1} wizard}
\newcommand{\field}[1]{\textit{#1} field}
\newcommand{\node}[1]{\textit{#1} node}
\newcommand{\folder}[1]{\textit{#1} folder}
\newcommand{\element}[1]{\textit{#1}}
\newcommand{\java}[1]{\texttt{#1}}
\newcommand{\filename}[1]{\texttt{#1}}

% --------------------------------------------------------------------------- %
% cross reference macros
% --------------------------------------------------------------------------- %
\newcommand{\prtlabel}[1]{\label{prt:#1}}
\newcommand{\applabel}[1]{\label{apx:#1}}
\newcommand{\chalabel}[1]{\label{cha:#1}}
\newcommand{\seclabel}[1]{\label{sec:#1}}
\newcommand{\lstlabel}[1]{lst:#1}
\newcommand{\figlabel}[1]{\label{fig:#1}}
\newcommand{\tablabel}[1]{\label{tab:#1}}

\newcommand{\prtref}[1]{Part~\ref{prt:#1}}
\newcommand{\appref}[1]{Appendix~\ref{apx:#1}}
\newcommand{\charef}[1]{Chapter~\ref{cha:#1}}
\newcommand{\secref}[1]{Section~\ref{sec:#1}}
\newcommand{\lstref}[1]{Listing~\ref{lst:#1}}
\newcommand{\figref}[1]{Figure~\ref{fig:#1}}
\newcommand{\tabref}[1]{Table~\ref{tab:#1}}

% --------------------------------------------------------------------------- %
% graphics paths
% --------------------------------------------------------------------------- %
\graphicspath{
  {figures/}
  {Introduction/figures/}
  {ScoutInstallation/figures/}  
}

%=============================================================================%

  \pagestyle{headings}
  \graphicspath{{figures/} {../figures/}}
  \begin{document}
  \sloppy
\fi


% --------------------------------------------------------------------------- %
\chapter{Preface}


% --------------------------------------------------------------------------- %
\section*{Who should read this Book?}

text needed

\begin{itemize}
\item You want to build a multi-user mobile application.
\item You want to build Java based business applications. 
\item You need to decide on the IT client side strategy of your organisation
\item You want to show off the great library you have built but have hardly time for the good looking UI that it deserves
\end{itemize}

pointer to \secref{whatshouldiread}.

% --------------------------------------------------------------------------- %
\section*{The Scout Community}

The community around Scout is its most valuable asset. That's why it deserves 
beeing mentionend as early as possible. Well, after you have decided that 
you possibly belong to the target audience of this book.

The Scout community is growing, friendly and active.
Here is a short list of resources that you may find useful:

\begin{itemize}
\item \url{http://www.eclipse.org/scout} is the main web site of Scout.
\item \url{http://wiki.eclipse.org/scout} the Scout wiki pages.
\item \url{http://www.eclipse.org/forums/eclipse.scout} the forum to ask questions related to Scout
\item \url{http://www.bsiag.com/scout} Scout blogging
\item \url{http://twitter.com/EclipseScout} Scout tweeting
\end{itemize}

\newpage

% --------------------------------------------------------------------------- %
\section*{This Book is Open and Free}

This book is an open book:

\begin{itemize}

\item \bf{Access is simple}. To download the book you 
do not need to register, look at ads or pay subscription fees. To cater for individual
preferences the book is avilable in various formats. Eventually, there should 
also be a printed version. And translations.

\item \bf{Reuse/rehash is simple}. The content of this book 
is released under the Creative Commons Attribution-ShareAlike license.
In short, you are allowed to freely share and adapt this book, as long as you 
respect the conditions of the license available at the following URL: \\
\url{http://creativecommons.org/licenses/by-sa/3.0/}

\item \bf{Contribution is simple}. Thats why the complete sources, setup, 
and continuous integration infrastucture is available on github.

\end{itemize}


% --------------------------------------------------------------------------- %
\section*{Thank You!}

We thank the various people and organisations that have contributed 
to this book. In particular, we thank xxx
Many good ideas regarding various aspects, such as book setup, etc. taken from 
numerous freely available resources on the web. 
Significant impact had ''Pharo by Example'', ...

We thank BSI Business Systems Integration AG for having made the bold move 
to open source the Eclipse Scout framework in the first place and its 
dedication to open standards in general and its continued support to further
evolve Scout and its growing community.

Last but not least we thank the Scout community for the many corrections and improvements 
made during the creation of the first version of The Eclipse Scout Book.

% --------------------------------------------------------------------------- %

\ifx\wholebook\relax\else
   \begin{thebibliography}{99}
  \addcontentsline{toc}{chapter}{Bibliography}
  
  \bibitem{lamport} L. Lamport. {\bf \LaTeX \ A Document Preparation System}
    Addison-Wesley, California 1986.

\end{thebibliography}

   \end{document}
\fi

% =========================================================================== %
