% =========================================================================== %
% Eclipse Basics
% =========================================================================== %

\ifx\wholebook\relax\else
  \documentclass[a4paper,10pt,twoside]{book}
  %=============================================================================%
% Common things, settings, packages to include
%=============================================================================%

\usepackage{graphicx}
\usepackage{color}
\usepackage{makeidx}
\usepackage{ifpdf}
\usepackage{verbatim}

\ifpdf
  \usepackage{pdfpages}
\fi

% --------------------------------------------------------------------------- %
% Setting up stuff depeding on output format
% --------------------------------------------------------------------------- %

\ifpdf
  % special settings for pdf mode
  \usepackage[colorlinks]{hyperref}
  \usepackage{courier}
  
  \hypersetup{
    colorlinks,
    linkcolor=darkblue,
    citecolor=darkblue,
    pdftitle={The Eclipse Scout Book},
    pdfauthor={The Scout Community},
    pdfkeywords={Enterprise Framework, Eclipse, Java, Client-Side, Rich Client, Web Client, Mobile},
    pdfsubject={Computer Science}
  }
  
  \usepackage{caption}
  \captionsetup{margin=10pt,font=small,labelfont=bf}
\else
  % special stuff for html mode
  \usepackage[tex4ht]{hyperref}
\fi

% --------------------------------------------------------------------------- %
% Setting up printing range
% --------------------------------------------------------------------------- %

\parindent 1cm
\parskip 0.2cm
\topmargin 0.2cm
\oddsidemargin 1cm
\evensidemargin 0.5cm
\textwidth 15cm
\textheight 21cm

% --------------------------------------------------------------------------- %
% Setting up paragraph formatting
% --------------------------------------------------------------------------- %

\setlength{\parindent}{20pt} 

% --------------------------------------------------------------------------- %
% Setting up listings
% --------------------------------------------------------------------------- %

\usepackage{listings}
 
\definecolor{darkviolet}{rgb}{0.5,0,0.4}
\definecolor{darkgreen}{rgb}{0,0.4,0.2} 
\definecolor{darkblue}{rgb}{0.1,0.1,0.9}
\definecolor{darkgrey}{rgb}{0.5,0.5,0.5}
\definecolor{lightblue}{rgb}{0.4,0.4,1}
\definecolor{lightgray}{rgb}{0.97,0.97,0.97}

\renewcommand{\lstlistlistingname}{List of Listings}

% general settings
\lstset{
  basicstyle=\small\ttfamily,
  columns=fullflexible,
  breaklines=true,
  breakindent=10pt,
  prebreak=\mbox{{\color{blue}\tiny$\searrow$}},
  postbreak=\mbox{{\color{blue}\tiny$\rightarrow$}},
  showstringspaces=false,
  backgroundcolor=\color{lightgray}
}

% settings for xml files
\lstdefinelanguage{xml}
{
  commentstyle=\color{darkgrey}\upshape,
  morestring=[b]",
  morestring=[s]{>}{<},
  morecomment=[s]{<?}{?>},
  stringstyle=\color{black},
  identifierstyle=\color{darkblue},
  keywordstyle=\color{cyan},
  morekeywords={xmlns,name,point,factory,class}% list your attributes here
}

% settings for ini files
\lstdefinelanguage{ini}
{
  morecomment=[f][\color{darkgrey}\upshape][0]\#, % # is comment iff it's the first char on the line
  stringstyle=\color{black}
}

% settings for console output
\lstdefinelanguage{console}
{
  morecomment=[l]{C:},
  commentstyle=\color{darkblue}
}

% default settings (for java files)
\lstset{
  language=Java,
  emphstyle=\color{red}\bfseries,
  keywordstyle=\color{darkviolet}\bfseries,
  commentstyle=\color{darkgreen},
  morecomment=[s][\color{lightblue}]{/**}{*/},
  stringstyle=\color{darkblue},
}

% --------------------------------------------------------------------------- %
% eclipse stuff macros
% --------------------------------------------------------------------------- %
\newcommand{\menu}[1]{\textsc{#1} menu}
\newcommand{\contextmenu}[1]{\textsc{#1} context menu}
\newcommand{\button}[1]{\textsc{#1} button}
\newcommand{\tab}[1]{\textsc{#1} tab}
\newcommand{\icon}[1]{\textsc{#1} icon}
\newcommand{\wizard}[1]{\textit{#1} wizard}
\newcommand{\field}[1]{\textit{#1} field}
\newcommand{\node}[1]{\textit{#1} node}
\newcommand{\folder}[1]{\textit{#1} folder}
\newcommand{\element}[1]{\textit{#1}}
\newcommand{\java}[1]{\texttt{#1}}
\newcommand{\filename}[1]{\texttt{#1}}

% --------------------------------------------------------------------------- %
% cross reference macros
% --------------------------------------------------------------------------- %
\newcommand{\prtlabel}[1]{\label{prt:#1}}
\newcommand{\applabel}[1]{\label{apx:#1}}
\newcommand{\chalabel}[1]{\label{cha:#1}}
\newcommand{\seclabel}[1]{\label{sec:#1}}
\newcommand{\lstlabel}[1]{lst:#1}
\newcommand{\figlabel}[1]{\label{fig:#1}}
\newcommand{\tablabel}[1]{\label{tab:#1}}

\newcommand{\prtref}[1]{Part~\ref{prt:#1}}
\newcommand{\appref}[1]{Appendix~\ref{apx:#1}}
\newcommand{\charef}[1]{Chapter~\ref{cha:#1}}
\newcommand{\secref}[1]{Section~\ref{sec:#1}}
\newcommand{\lstref}[1]{Listing~\ref{lst:#1}}
\newcommand{\figref}[1]{Figure~\ref{fig:#1}}
\newcommand{\tabref}[1]{Table~\ref{tab:#1}}

% --------------------------------------------------------------------------- %
% graphics paths
% --------------------------------------------------------------------------- %
\graphicspath{
  {figures/}
  {Introduction/figures/}
  {ScoutInstallation/figures/}  
}

%=============================================================================%

  \pagestyle{headings}
  \graphicspath{{figures/} {../figures/}}
  \begin{document}
  \sloppy
\fi


% --------------------------------------------------------------------------- %
\chapter{Eclipse Basics}
\applabel{eclipse_basics}

% --------------------------------------------------------------------------- %
\section{Eclipse as an IDE}
\applabel{eclipse_ide}
\applabel{eclipse_perspective}

To get a good introduction into working with Eclipse as an IDE we highly recommend the excellent Eclipse IDE tutorial by L. Vogel \url{http://www.vogella.com/articles/Eclipse/article.html}.

An Eclipse IDE perspective contains the visual elements and the arrangement of those elements to support a specific development task within the Eclipse IDE. 
Perspectives relevant to the development of Scout applications are the Scout perspective, the Java perspective, the Debug perspective, and many others. 
To open a perspective available in the Eclipse IDE, the \button{Open Perspective} or the \menu{Window $\rightarrow$ Open Perspective $\rightarrow$ Other...} can be used. 

\begin{figure}
\includegraphics[width=14cm]{eclipse_ide_parts.png} 
\caption{The Eclipse IDE with the Scout perspective. The colors indicate the different elements. View parts (blue), editor parts (green) and perspective buttons (purple). }
\figlabel{eclipse_ide_parts}
\end{figure}

\figref{eclipse_ide_parts} provides a screenshot of the Eclipse Scout perspective indicating the corresponding perspective button and the main view parts and editor parts involved. 
Using drag and drop, views and editors can be freely moved around in the Eclipse IDE to suit the developer's needs.
Perspectives can be further individualized using the \menu{Window $\rightarrow$ Customize Perspective...}. 
Here, the visibility of the toolbar items and menu entries can be defined. 
Once a suitable layout of all desired elements has been defined, this organisation may be saved as a personal perspective using the Eclipse IDE \menu{Window $\rightarrow$ Save Perspective As...}.

In case a customizing step does not turn out as intended, with the \menu{Window $\rightarrow$ Reset Perspective...} is always possible to go back to the last saved state of the current perspective.
  
\ifx\wholebook\relax\else
   \begin{thebibliography}{99}
  \addcontentsline{toc}{chapter}{Bibliography}
  
  \bibitem{lamport} L. Lamport. {\bf \LaTeX \ A Document Preparation System}
    Addison-Wesley, California 1986.

\end{thebibliography}

   \end{document}
\fi

% =========================================================================== %
